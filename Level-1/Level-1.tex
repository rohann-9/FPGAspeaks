\documentclass{article}
\usepackage{graphicx} % Required for inserting images
\title{\huge{\textbf{FPGAspeaks}}}
\author{\textbf{Level-1}\\  \textbf{Samuel Rohan}}
\date{September 2023}
\usepackage{geometry}
\geometry{a4paper,margin=2cm}
\usepackage{hyperref}
\begin{document}
\maketitle
\tableofcontents
\section{Introduction}
This Level will give a brief intro of the all the Basic logic gates which will be used in our upcoming levels.
\section{Tools/Requirements}
\textbf{Xilinx Vivado}
\section{Basic Logic Gates}
\begin{itemize}
    \item \textbf{OR Gate:}
    A logic gate that produces a HIGH output when one or more inputs are HIGH.
    \item \textbf{AND Gate:}
    A logic gate that produces a HIGH output only when all the inputs are HIGH.
    \item \textbf{NOT Gate:}
    A logic gate that produces a HIGH output when input is LOW and Vice versa.
    \item \textbf{XOR Gate:}
    A logic gate that produces a HIGH output only when inputs are NOT equal.
    \item \textbf{NAND Gate:}
    A logic gate that produces a LOW output only when all the inputs are HIGH.
    \item \textbf{NOR Gate:}
    A logic gate that produces a LOW output when one or more inputs are HIGH.
    \item \textbf{XNOR Gate:}
    A logic gate that produces a HIGH output only when the inputs are equal.
\end{itemize}
\section{What have I done in this Level?}
I have implemented the verilog programs of all the Basic logic gates using Xilinx 
Vivado.
\section{Links}
\href{https://github.com/rohann-9/FPGAspeaks/tree/main/Level-1}{Link to my Github}
\end{document}
